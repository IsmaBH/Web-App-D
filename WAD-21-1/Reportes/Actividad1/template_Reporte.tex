\documentclass{article}
\usepackage{graphicx}
\begin{document}
	\begin{titlepage}
		\centering
		{\bfseries\LARGE Instituto Politécnico Nacional \par}
		\vspace{1cm}
		{\scshape\Large Escuela Superior de Computo \par}
		\vspace{3cm}
		{\scshape\Huge Instalación de GIT \par}
		\vspace{3cm}
		{\itshape\Large Web App Development \par}
		\vfill
		{\Large Profesor: \par}
		{\Large M. en C. José Asunción Enríquez Zárate \par}
		\vspace{1cm}
		{\Large Autor: \par}
		{\Large Barón Hernández Diego Ismael \par}
		\vfill
		{\Large 3CM9 \par}
	\end{titlepage}
	\newpage
	\tableofcontents
	\newpage
	\section{Introducción}
		Git es un sistema de control de versiones distribuido de código abierto y gratuito diseñado para manejar todo, desde proyectos pequeños hasta muy grandes, con velocidad y eficiencia.\\
		Git es fácil de aprender y ocupa poco espacio con un rendimiento increíblemente rápido. Supera a las herramientas SCM como Subversion, CVS, Perforce y ClearCase con características como bifurcaciones locales económicas, áreas de preparación convenientes y múltiples flujos de trabajo.
	\section{Conceptos}
		\subsection{Control de versiones}
			Se llama control de versiones a la gestión de los diversos cambios que se realizan sobre los elementos de algún producto o una configuración del mismo. Una versión, revisión o edición de un producto, es el estado en el que se encuentra el mismo en un momento dado de su desarrollo o modificación.
	\section{Desarrollo}
		La instalación se llevo acabo en un entorno de desarrollo UNIX/Ubuntu.
		\subsection{Instalación de GIT}
		La instalación de git es muy sencilla solo se requiere tener una terminal abierta y teclear el siguiente comando:
		\begin{figure}[h]
			\centering
			\includegraphics[scale=0.5]{/home/diego/Imágenes/Actividad1/git_cmd.png}
			\caption{Comando de Instalación de GIT}
		\end{figure}
		\\Al termino de este proceso ya tendremos instalado git y solo deberemos configurar las credenciales de nuestro sistema.
		\subsection{Configuración de credenciales}
		Ahora configuramos el correo y el nombre de usuario con los siguientes comandos:
		\begin{figure}[h]
			\centering
			\includegraphics[scale=0.5]{/home/diego/Imágenes/Actividad1/config_cmd.png}
			\caption{Comandos de configuración de git}
		\end{figure}
		\\Y con esto ya tenemos nuestra instalación de GIT lista para trabajar de forma básica.
	\section{Conclusiones}
	Esta es una herramienta muy útil debido a que nos permite tener siempre un control de nuestros trabajos o proyectos, ya sea que surja algún imprevisto y perdamos los datos de una pc no tendríamos que preocuparnos por recuperar el proyecto en el que trabajábamos ya que estaría disponible en el servidor.
	\section{Referencias}
	https://git-scm.com/
\end{document} 