\documentclass{article}
\begin{document}
	\begin{titlepage}
		\centering
		{\bfseries\LARGE Instituto Politécnico Nacional \par}
		\vspace{1cm}
		{\scshape\Large Escuela Superior de Computo \par}
		\vspace{3cm}
		{\scshape\Huge Instalación de Apache-Tomcat \par}
		\vspace{3cm}
		{\itshape\Large Web App Development \par}
		\vfill
		{\Large Profesor: \par}
		{\Large M. en C. José Asunción Enríquez Zárate \par}
		\vspace{1cm}
		{\Large Autor: \par}
		{\Large Barón Hernández Diego Ismael \par}
		\vfill
		{\Large 3CM9 \par}
	\end{titlepage}
	\newpage
	\tableofcontents
	\newpage
	\section{Introducción}
		Apache Tomcat es un servidor web y contenedor de servlets que se utiliza para presentar aplicaciones Java. Tomcat es una implementación de código abierto de las tecnologías Java Servlet y JavaServer Pages publicada por la Apache Software Foundation.
	\section{Conceptos}
		\subsection{Variables de entorno}
			Normalmente, son valores que hacen referencia a archivos, directorios y funciones comunes del sistema cuya ruta concreta puede variar, pero que otros programas necesitan poder conocer tal como es el caso de de nuestro servidor Tomcat con la instalación de Java.
		\subsection{CATALINA HOME}
			Esta es una variable de entorno que señala donde esta la instalación principal de nuestro servidor Tomcat, también hace referencia a la carpeta donde se pueden encontrar los .sh para levantar el servidor, archivos de configuración y usuarios.
		\subsection{JAVA HOME}
			Es la variable de entorno que le dice a nuestro servidor que versión de Java se usara así como donde esta para que se pueda ejecutar el código Java que pongamos en el servidor.
	\section{Desarrollo}
		Esta guiá se realizo en un entorno UNIX/Ubuntu por lo que habrá alguno términos que son particulares de la distribución.
		\subsection{Instalación de Java}
		
\end{document} 