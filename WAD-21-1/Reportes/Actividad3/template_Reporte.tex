\documentclass{article}
\usepackage{graphicx}
\begin{document}
	\begin{titlepage}
		\centering
		{\bfseries\LARGE Instituto Politécnico Nacional \par}
		\vspace{1cm}
		{\scshape\Large Escuela Superior de Computo \par}
		\vspace{3cm}
		{\scshape\Huge Instalar Heroku CLI \par}
		\vspace{3cm}
		{\itshape\Large Web App Development \par}
		\vfill
		{\Large Profesor: \par}
		{\Large M. en C. José Asunción Enríquez Zárate \par}
		\vspace{1cm}
		{\Large Autor: \par}
		{\Large Barón Hernández Diego Ismael \par}
		\vfill
		{\Large 3CM9 \par}
	\end{titlepage}
	\newpage
	\tableofcontents
	\newpage
	\section{Introducción}
		La interfaz de línea de comandos de Heroku (CLI) facilita la creación y administración de sus aplicaciones de Heroku directamente desde el terminal. Es una parte esencial del uso de Heroku.
	\section{Conceptos}
		\subsection{Cliente}
			Es un sistema local que facilita el manejo de, en este caso, la aplicación de Heroku desde una terminal UNIX.
	\section{Desarrollo}
		Esta guiá se realizo en un entorno UNIX/Ubuntu por lo que habrá alguno términos que son particulares de la distribución.
		\subsection{Instalación de Heroku CLI}
		Lo primero a realizar es la instalación mediante el siguiente comando:
		\begin{figure}[h]
			\centering
			\includegraphics[scale=0.5]{/home/diego/Imágenes/Actividad3/instalacion_cmd.png}
			\caption{Comando de Instalación}
		\end{figure}
		Esto al terminar nos mostrara el siguiente mensaje en la terminal:
		\begin{figure}[h]
			\centering
			\includegraphics[scale=0.5]{/home/diego/Imágenes/Actividad3/instalacionc_cmd.png}
			\caption{Resultado de instalación}
		\end{figure}
		\subsection{Verificar la instalación}
		Ahora debemos comprobar que se haya instalado correctamente y la versión mediante el comando:
		\begin{figure}[h]
			\centering
			\includegraphics[scale=0.5]{/home/diego/Imágenes/Actividad3/version_cmd.png}
			\caption{Comando de version}
		\end{figure}
		\\Lo que debe darnos una respuesta en la terminal como la que sigue:
		\begin{figure}[h]
			\centering
			\includegraphics[scale=0.5]{/home/diego/Imágenes/Actividad3/vresponse_cmd.png}
			\caption{Respuesta del comando version}
		\end{figure}
		\subsection{Credenciales}
		Ahora debemos configurar nuestras credenciales de acceso lo haremos mediante consola con el comando:
		\begin{figure}[h]
			\centering
			\includegraphics[scale=0.5]{/home/diego/Imágenes/Actividad3/credenciales_cmd.png}
			\caption{Comando de credenciales}
		\end{figure}
		\\En nuestro caso el resultado de este comando nos queda de la siguiente manera:
		\begin{figure}[h]
			\centering
			\includegraphics[scale=0.3]{/home/diego/Imágenes/Actividad3/rcredenciales_cmd.png}
			\caption{Comando de descarga}
		\end{figure}
	\section{Conclusiones}
	Dado que trabaja en conjunto con el cliente de GIT es logico concluir que nos ayudara bastante para el desarrollo de nuestras aplicaciones de manera local para después publicarlas.
	\section{Referencias}
	https://devcenter.heroku.com/articles/heroku-cli
\end{document} 